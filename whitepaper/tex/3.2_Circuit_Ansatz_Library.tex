Ansatz is a common component in VQA. It contains a series of parameters that can be trained and optimized. There may be many different Ansatz for the same VQA algorithm, and they have different effects. Some specially designed Ansatz circuit may have superior performance or expression ability. Therefore, selecting or designing a suitable Ansatz is an important part of the success of a VQA algorithm.

The $algorithm$ module in \MindQuantum contains many commonly used Ansatz, as shown below.

\begin{itemize}
    \item $IQPEncoding$: The IQPEncoding ansatz provides an ansatz to encode classical data into quantum state.

    \item $HardwareEfficientAnsatz$:It is a kind of ansatz that can be easily implement on quantum chip. The hardware efficient is constructed by a layer of single qubit rotation gate and a layer of two qubits entanglement gate. The single qubit rotation gate layer is constructed by one or several rotation gate that act on every qubit. The two qubits entanglement gate layer is constructed by CNOT, CZ, Rxx, Ryy, Rzz, etc. acting on entangle mapping.

    \item $StronglyEntangling$: It is usually used in quantum machine learning. Its advantage is that it can reach 'wide
corners of the Hilbert space'\cite{Schuld_2020}, so it is considered to be very suitable as ansatz of classifier.

    \item $QAOA ansatz$:
    \item $VQE Ansatz$:
\end{itemize}

\MindQuantum also implemented 19 ansatz for quantum classical hybrid algorithms, all based on this article analyzing the expressibility and entanglement capabilities of various parameterized circuits.\cite{Sim_2019}

In addition, there are some quantum chemistry-related generators in \MindQuantum that can generate corresponding VQE ansatz.

\begin{itemize}
    \item $generate\_uccsd$: Generate a uccsd quantum circuit based on a molecular data generated by Openfermion.
    \item $quccsd\_generator$: Generate qubit-UCCSD (qUCCSD) ansatz using qubit-excitation operators.
    \item $uccsd0\_singlet\_generator$: Generate UCCSD operators using CCD0 ansatz for molecular systems.
    \item $uccsd\_singlet\_generator$: Create a singlet UCCSD generator for a system with n electrons.
\end{itemize}


Users can also design their own Ansatz by inheriting the Ansatz base class and implementing the $implement()$ method.

\begin{lstlisting}
from mindquantum.core.gates import RX, RY
from mindquantum.algorithm.nisq import Ansatz

class MyAnsatz(Ansatz):
    def __init__(self, n_qubits, depth):
        """Initialize my ansatz."""
        Ansatz.__init__(self, 'MyAnsatz', n_qubits, depth)

    def _implement(self, depth):
        """Implement of my ansatz."""
        for i in range(depth):
            for j in range(self.n_qubits):
                self._circuit += RX(f'a_{i}').on(j)
                self._circuit += RY(f'b_{i}').on(j)

ansatz_circuit = MyAnsatz(3, 2).circuit
\end{lstlisting}