% \begin{lstlisting}
% 理论与实现:
% 1. 费米子
% 2. 玻色子
% 3. mq中如何构造这些算子

% 支持功能:
% 1. 运算
%   a. 算子间相乘
%   b. 加上一个数
% 2. Transfrom
% \end{lstlisting}
View demo code of this section: \democode{02}{2.4}

Simulating many-body physical systems was the initial application scenario proposed by Richard Feynman for quantum computers. Effective simulations of such systems can aid us in understanding and designing new materials. Among the various many-body physics models, we are particularly interested in the Heisenberg model and the Fermi-Hubbard model. The Hamiltonians for these models can be expressed as follows:

\begin{align*}
    H_\text{Heisenberg}    & = -J\sum_{<i,j>}\sigma_i\otimes \sigma_j-h\sum_i\sigma_i                                             \\
    H_\text{Fermi-Hubbard} & = -\sum_{<i,j>,\sigma}\left(a_{i\sigma}^\dagger a_{j\sigma} + a_{j\sigma}^\dagger a_{i\sigma}\right) \\
                           & + U\sum_i n_{i\uparrow}n_{i\downarrow}
\end{align*}

In \MindQuantum, it is straightforward for us to construct those hamiltonian with the help of \QubitOperator and \FermionOperator.

\subsubsection{Qubit Operators}

In Heisenberg model, $\sigma_i$ represent the pauli operator. The matrix form of pauli operator is:

\begin{align*}
    \sigma_X & = \begin{pmatrix}
                     0 & 1 \\
                     1 & 0
                 \end{pmatrix}, \sigma_Y = \begin{pmatrix}
                                               0 & -i \\
                                               i & 0
                                           \end{pmatrix} \\
    \sigma_I & =\begin{pmatrix}
                    1 & 0 \\
                    0 & 1
                \end{pmatrix}, \sigma_Z = \begin{pmatrix}
                                              1 & 0  \\
                                              0 & -1
                                          \end{pmatrix}
\end{align*}
In \MindQuantum, \QubitOperator is used to build this kind of operator. Given a pauli operator $\sigma_{X,3}\otimes \sigma_{Y,1}\otimes \sigma_{Z,0}=X_3 Y_1 Z_0$, which means apply pauli $Z$, pauli $Y$ and pauli $X$ on qubit 3, 1 and 0,  we can easily construct it with:

\begin{lstlisting}
from mindquantum.core.operators import QubitOperator

ops = QubitOperator('Z0 Y1 X3')
\end{lstlisting}
Please note that since $[Z_0, Y_1]= [Z_0,X_3] = [Y_1, X_3]=0$, so the order of pauli word in the pauli string does not matters.

\QubitOperator also support arithmetic operation, which can help you to build more complex operators:

\begin{lstlisting}
from mindquantum.core.operators import QubitOperator
from mindquantum.core.parameterresolver import ParameterResolver as PR

op1 = QubitOperator('X0')
op2 = QubitOperator('Z1', 'a')
op3 = QubitOperator('Y1')
op4 = 2 * op1 * op2 + op3 * PR('b')
print(op4)
print(op4.subs({'a':1, 'b':2}).matrix().toarray())
\end{lstlisting}
The output is:
\begin{lstlisting}
2*a [X0 Z1] +
  b [Y1]

[[ 0.+0.j  2.+0.j  0.-2.j  0.+0.j]
 [ 2.+0.j  0.+0.j  0.+0.j  0.-2.j]
 [ 0.+2.j  0.+0.j  0.+0.j -2.+0.j]
 [ 0.+0.j  0.+2.j -2.+0.j  0.+0.j]]
\end{lstlisting}
In the last line, we use \code{.subs} to set the value of parameters and get the csr format sparse matrix with \code{.matrix}.

\subsubsection{Fermion Operators}

In Fermi-Hubbard model, $a_i^\dagger$ and $a_i$ are creation and annihilation operators of fermion. Different from pauli operator, the fermion operator follows anti-commutation relation:
\begin{align*}
    \{a_i, a_j^\dagger\} & = a_ia_j^\dagger + a_j^\dagger a_i = \delta_{ij} \\
    \{a_i, a_j\}         & = \{a_i^\dagger, a_j^\dagger\}=0
\end{align*}
In qubit system, the creation and annihilation operators acting on state $\ket{0}$ and $\ket{1}$ follows:
\begin{align*}
    a\ket{0}          & =0,        & a\ket{1}          & =\ket{0} \\
    a^\dagger \ket{0} & = \ket{1}, & a^\dagger \ket{1} & =0
\end{align*}

The matrix form of creation and annihilation would be:

\begin{align*}
    a=\begin{pmatrix}
          0 & 1 \\
          0 & 0
      \end{pmatrix},
    a^\dagger=\begin{pmatrix}
                  0 & 0 \\
                  1 & 0
              \end{pmatrix}
\end{align*}
In \MindQuantum, we use \FermionOperator to construct fermion operator. Suppose we first create a state on qubit 1 and annihilate a state on qubit 0, the operator will be $a_1^\dagger\otimes a_0$, and you can build it with:
\begin{lstlisting}
from mindquantum.core.operators import FermionOperator
op1 = FermionOperator('1^ 0')
\end{lstlisting}
In the fermion string, we use \verb|^| to represent $\dagger$ and the number to tell us on which qubit that we act. The arithmetic operation of \FermionOperator is very similar with \QubitOperator, and we will not discuss it more.


\subsubsection{Operator Functions}

Mindquantum also supplies a number of advanced functions for Operators. Here are some show case:
\begin{itemize}
    \item \methodcommutator{op1}{op2} : Calculate the commutator of two operators.
          \begin{lstlisting}
from mindquantum.core.operators import QubitOperator, FermionOperator, commutator
qub_op1 = QubitOperator("X1 Y2")
qub_op2 = QubitOperator("X1 Z2")
commutator(qub_op1, qub_op1)  # 0
commutator(qub_op1, qub_op2)  # (2j) [X2]
    \end{lstlisting}
    \item \methodcountqubits{op1} : Count the number of qubits before deleting unused qubits.
          \begin{lstlisting}
from mindquantum.core.operators import QubitOperator,FermionOperator, count_qubits
qubit_op = QubitOperator("X1 Y2")
count_qubits(qubit_op)  # 3
fer_op = FermionOperator("1^")
count_qubits(fer_op)  # 2
    \end{lstlisting}
    \item \methodhermitianconj{op1} : Get the hermitian conjugation of given operator.
          \begin{lstlisting}
from mindquantum.core.operators import FermionOperator, hermitian_conjugated
fer_op = FermionOperator("1^ 3")
hermitian_conjugated(fer_op)
    \end{lstlisting}
\end{itemize}
