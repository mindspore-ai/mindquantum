View demo code of this section: \democode{02}{2.4}

Simulating many-body physical systems was the initial application scenario proposed by Richard Feynman for quantum computers. Effective simulations of such systems can aid us in understanding and designing new materials. Among the various many-body physics models, we are particularly interested in the Heisenberg model and the Fermi-Hubbard model. The Hamiltonians for these models can be expressed as follows:

\begin{align*}
    H_\text{Heisenberg}    & = -J\sum_{<i,j>}\sigma_i\otimes \sigma_j-h\sum_i\sigma_i,                                            \\
    H_\text{Fermi-Hubbard} & = -\sum_{<i,j>,\sigma}\left(a_{i\sigma}^\dagger a_{j\sigma} + a_{j\sigma}^\dagger a_{i\sigma}\right) \\
                           & + U\sum_i n_{i\uparrow}n_{i\downarrow}.
\end{align*}

In \MindQuantum, it is straightforward for us to construct those Hamiltonian with the help of \QubitOperator and \FermionOperator.

\subsubsection{Qubit Operators}

In Heisenberg model, $\sigma_i$ represent the Pauli operator. The matrix form of Pauli operator is:

\begin{align*}
    \sigma_X & = \begin{pmatrix}
        0 & 1 \\
        1 & 0
    \end{pmatrix}, \sigma_Y = \begin{pmatrix}
        0 & -i \\
        i & 0
    \end{pmatrix}, \\
    \sigma_I & =\begin{pmatrix}
        1 & 0 \\
        0 & 1
    \end{pmatrix}, \sigma_Z = \begin{pmatrix}
        1 & 0  \\
        0 & -1
    \end{pmatrix}.
\end{align*}
In \MindQuantum, \QubitOperator is used to build this kind of operator. Given a Pauli operator $\sigma_{X,3}\otimes \sigma_{Y,1}\otimes \sigma_{Z,0}=X_3 Y_1 Z_0$, which means apply Pauli $Z$, Pauli $Y$ and Pauli $X$ on qubit 3, 1 and 0,  we can easily construct it with:

\begin{lstlisting}
from mindquantum.core.operators import QubitOperator

ops = QubitOperator('Z0 Y1 X3')
\end{lstlisting}
Please note that since $[Z_0, Y_1]= [Z_0,X_3] = [Y_1, X_3]=0$, so the order of Pauli word in the Pauli string does not matter.

\QubitOperator also support arithmetic operation, which can help you to build more complex operators:

\begin{lstlisting}
from mindquantum.core.operators import QubitOperator
from mindquantum.core.parameterresolver import ParameterResolver as PR

op1 = QubitOperator('X0')
op2 = QubitOperator('Z1', 'a')
op3 = QubitOperator('Y1')
op4 = 2 * op1 * op2 + op3 * PR('b')
print(op4)
print(op4.subs({'a':1, 'b':2}).matrix().toarray())
\end{lstlisting}
The output is:
\begin{lstlisting}
2*a [X0 Z1] +
  b [Y1]

[[ 0.+0.j  2.+0.j  0.-2.j  0.+0.j]
 [ 2.+0.j  0.+0.j  0.+0.j  0.-2.j]
 [ 0.+2.j  0.+0.j  0.+0.j -2.+0.j]
 [ 0.+0.j  0.+2.j -2.+0.j  0.+0.j]]
\end{lstlisting}
In the last line, we use \code{.subs} to set the value of parameters and get the csr format sparse matrix with \code{.matrix}.

\subsubsection{Fermion Operators}

In Fermi-Hubbard model, $a_i^\dagger$ and $a_i$ are creation and annihilation operators of fermion. Different from Pauli operator, the fermion operator follows anti-commutation relation:
\begin{align*}
    \{a_i, a_j^\dagger\} & = a_ia_j^\dagger + a_j^\dagger a_i = \delta_{ij}, \\
    \{a_i, a_j\}         & = \{a_i^\dagger, a_j^\dagger\}=0.
\end{align*}
In qubit system, the creation and annihilation operators acting on state $\ket{0}$ and $\ket{1}$ follows:
\begin{align*}
    a\ket{0}          & =0,        & a\ket{1}          & =\ket{0}, \\
    a^\dagger \ket{0} & = \ket{1}, & a^\dagger \ket{1} & =0.
\end{align*}

The matrix form of creation and annihilation would be:

\begin{align*}
    a=\begin{pmatrix}
        0 & 1 \\
        0 & 0
    \end{pmatrix},
    a^\dagger=\begin{pmatrix}
        0 & 0 \\
        1 & 0
    \end{pmatrix}.
\end{align*}
In \MindQuantum, we use \FermionOperator to construct fermion operator. Suppose we first create a state on qubit 1 and annihilate a state on qubit 0, the operator will be $a_1^\dagger\otimes a_0$, and you can build it with:
\begin{lstlisting}
from mindquantum.core.operators import FermionOperator
op1 = FermionOperator('1^ 0')
\end{lstlisting}
In the fermion string, we use \verb|^| to represent $\dagger$ and the number to tell us on which qubit that we act. The arithmetic operation of \FermionOperator is very similar with \QubitOperator, and we will not discuss it more.


\subsubsection{Operator Functions}

\MindQuantum\ also supplies a number of advanced functions for operators. Here are some showcases:

- \methodcommutator{op1}{op2} : Calculate the commutator of two operators.

\begin{lstlisting}
from mindquantum.core.operators import QubitOperator, FermionOperator, commutator
qub_op1 = QubitOperator("X1 Y2")
qub_op2 = QubitOperator("X1 Z2")
commutator(qub_op1, qub_op1)  # 0
commutator(qub_op1, qub_op2)  # (2j) [X2]
\end{lstlisting}

- \methodcountqubits{op1} : Count the number of qubits before deleting unused qubits.

\begin{lstlisting}
from mindquantum.core.operators import QubitOperator,FermionOperator, count_qubits
qubit_op = QubitOperator("X1 Y2")
count_qubits(qubit_op)  # 3
fer_op = FermionOperator("1^")
count_qubits(fer_op)  # 2
\end{lstlisting}


- \methodhermitianconj{op1} : Get the hermitian conjugation of given operator.

\begin{lstlisting}
from mindquantum.core.operators import FermionOperator, hermitian_conjugated
fer_op = FermionOperator("1^ 3")
hermitian_conjugated(fer_op)
\end{lstlisting}

\subsubsection{Transformation}
Quantum simulation of fermionic systems presents a specific challenge. To effectively reduce the resources used to simulate fermion Hamiltonian on quantum hardware, we can simulate fermions with qubits, which involves the conversion between the fermionic Hamiltonian and the qubit Hamiltonian. \MindQuantum\ supplies the module \code{mindquantum.algorithm.nisq.Transform} for transformation between fermions (\FermionOperator) and bosons (\QubitOperator). The following functions are provided:
\begin{itemize}
    \item \methodjordanwigner: Apply the Jordan-Wigner transformation, which maps fermionic annihilation operators (fermions) to qubits (bosons) via:
          \begin{equation}
              \begin{split}
                  a_j^{\dagger} \to {\sigma}_j^{-} X \prod_{i=0}^{j-1} {\sigma}^Z_i \\
                  a_j \to {\sigma}_j^{+} X \prod_{i=0}^{j-1} {\sigma}^Z_i,
              \end{split}
          \end{equation}
          where ${\sigma}_j^{+} = {\sigma}_j^X+i{\sigma}_j^Y$, ${\sigma}_j^{-} = {\sigma}_j^X-i{\sigma}_j^Y$, are the spin-up operator and spin-down operator respectively. The operator $\prod_{i=0}^{p-1} {{\sigma}^Z_i}$ is a “parity operator” with eigenvalues ±1, corresponding to eigenstates for which the subset of bits with index less than or equal to $i$ has even or odd parity, respectively. This transformation can retain the locality of the initial occupation number. The problem with this method is that as a consequence of the non-locality of the parity operator, the number of extra qubit operations required to simulate a single fermionic operator scales as $O(n)$.
    \item \methodparity: Apply the parity transformation, which uses qubit $j$ to store the $parity$ of all occupied orbitals up to orbital $j$. That is, we could let qubit $j$ store $p_j = (\sum_{i=0}^j f_i) mod 2$. This encoding of fermionic states in qubit states is called the $parity$ basis. In terms of these vectors (e.g. $(j_7,...,j_1,j_0)^T$), which is the occupation number basis states equivalent to, the map to the parity basis is given by:
          \begin{equation}
              p_i = \sum_j {[\pi_n]_{ij} f_j},
          \end{equation}
          where $n$ is the number of orbitals, $\pi_n$ is the ($n \times n$) matrix defined below:
          \begin{equation}
              [\pi_n]_{ij} =
              \begin{cases}
                  1 & i<j     \\
                  0 & i\geq j \\
              \end{cases}.
          \end{equation}
          The representations of the creation and annihilation operators in the parity basis are then:
          \begin{equation}
              \begin{split}
                  a_j^{\dagger} \to \frac{1}{2} (\prod_{i=j+1}^n ({\sigma}^X_i X))({\sigma}^X_j - i{\sigma}^Y_j) X {\sigma}^Z_{j-1} \\
                  a_j \to \frac{1}{2} (\prod_{i=j+1}^n ({\sigma}^X_i X))({\sigma}^X_j + i{\sigma}^Y_j) X {\sigma}^Z_{j-1}.
              \end{split}
          \end{equation}
    \item \methodbravyikitaev \cite{bravyi2002fermionic, fenwick1994new}: The previous two approaches are dual in the way that they store the information required to simulate fermionic operators with qubits. With the occupation number basis and its associated Jordan-Wigner transformation, the occupation information is stored locally but the parity information is non-local, whereas in the parity basis method and its corresponding operator transformation, the parity information is stored locally but the occupation information is non-local.
          The Bravyi-Kitaev transformation is a middle ground, it balances the locality of occupation and parity information for improved simulation efficiency. In this scheme, qubits store the parity of a set of $2^x$ orbitals, where $x \geq 0$. A qubit of index $j$ always stores orbital $j$. For even values of $j$, this is the only orbital that it stores, but for odd values of $j$, it also stores a certain set of adjacent orbitals with index less than $j$. The map from the occupation number basis to the Bravyi-Kitaev basis is:
          \begin{equation}
              b_i = \sum_j {[\beta_n]_{ij} f_j},
          \end{equation}
          where $n$ is the number of orbitals, $\beta_n$ is an ($n \times n$) square matrix. See \cite{seeley2012bravyi} for a detailed explanation.
    \item \methodbravyikitaevsuperfast: A fast version of $bravyi\_kitaev()$, which can perform the Bravyi-Kitaev transformation at a faster speed \cite{setia2018bravyi}. Note that only Hermitian operators such as the following can be transformed:
          \begin{equation}
              C + \sum_{p,q} h_{p,q} a_p^{\dagger} a_q + \sum_{p,q,r,s} h_{p,q,r,s} a_p^{\dagger} a_q^{\dagger} a_r a_s,
          \end{equation}
          where $C$ is a constant.
    \item \methodreversedjordanwigner: Apply inverse transformation of Jordan-Wigner, which will transform $QubitOperator$ to $FermionOperator$.
    \item \methodternarytree: Apply the Ternary-Tree transformation. This function is based on \cite{jiang2020optimal}.
\end{itemize}
In the code below, we use the function $Transform()$ to transform $FermionOperator$ to $QubitOperator$.
\begin{lstlisting}
from mindquantum.core.operators import FermionOperator
from mindquantum.algorithm.nisq import Transform
op1 = FermionOperator('1^')
op_transform = Transform(op1)
op_transform.jordan_wigner()
# 0.5 [Z0 X1] +
# -0.5j [Z0 Y1]
op_transform.parity()
# 0.5 [Z0 X1] +
# -0.5j [Y1]
op_transform.bravyi_kitaev()
# 0.5 [Z0 X1] +
# -0.5j [Y1]
op2 = FermionOperator('1^', 'a')
Transform(op2).jordan_wigner()
# 0.5*a [Z0 X1] +
# -0.5*I*a [Z0 Y1]
\end{lstlisting}