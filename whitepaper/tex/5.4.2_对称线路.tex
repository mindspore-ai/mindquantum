The study of variational quantum algorithms has largely been focused on identifying the ground state of intricate many-body systems. In this context, the Variational Quantum Eigensolver (VQE) algorithm has emerged as a potent tool, designed to target the ground state of a many-body system by minimizing average energy. However, several critical phenomena in physics, including topological phases, necessitate knowledge of several low-energy eigenstates, not just the ground state. Therefore, the generalization of VQE to higher energy eigenstates is highly important. 

The weighted SSVQE provides an alternative method to generate all the $k$ lowest energy eigenstates of a given Hamiltonian $H$~\cite{?}. This method utilizes a set of $k$ orthogonal initial states, denoted as $\{|\phi_{i}\rangle\}_{i=1}^{k}$ (where $\langle \phi_{i} | \phi_{j} \rangle = \delta_{ij}$), as the input of a single parameterized quantum circuit, described by the unitary operator $U(\vec{\theta})$. Given that the initial states are orthogonal, the outputs $U(\vec{\theta})| \phi_{j} \rangle$, generated by the same circuit, maintain orthogonality. In the weighted SSVQE, the objective is to minimize the cost function  
\begin{equation}
    \mathrm{cost} = \sum_{i=1}^{k} w_{i} \langle \phi_{i}| U^{\dagger}(\vec{\theta}) H U(\vec{\theta}) | \phi_{i} \rangle
    \label{ssvqe_cost}
\end{equation}
where $w_1 > w_2 > \cdots > w_k$ are real positive numbers. Minimizing the cost function in Eq.~\eqref{ssvqe_cost} produces all the $k$ lowest energy eigenstates such that $|E_{i}\rangle = U(\vec{\theta}^{*})|\phi_{i}\rangle$. 
A notable advantage of the weighted SSVQE is that it delivers all the $k$ lowest energy eigenstates through a single optimization process, without requiring any overlap of quantum states. However, the algorithm becomes more resource-demanding as the number of target eigenstates increases. 

Symmetry is one of the most profound concepts in physics, especially in quantum mechanics. A majority of physical systems exhibit various types of symmetries that can be accurately described mathematically. The VQE algorithm can also significantly benefit from the integration of these symmetries. There are two ways to incorporate symmetries in the VQE algorithms: (i) designing the circuit to naturally generate the quantum states with the relevant symmetry~\cite{?}, and (ii) adding extra terms to the cost function to penalize the quantum states without the relevant symmetry~\cite{?}. 
  

