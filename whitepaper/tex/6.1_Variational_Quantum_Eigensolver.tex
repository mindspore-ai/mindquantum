% \begin{lstlisting}
% 设计原理:粒子数和自选数守恒

% 性能与优势:
% 1. 简单罗列数据,比mq快多少
% 2. 强调对内存的压缩能力
% \end{lstlisting}

Simulating molecule system with high accuracy in VQE is both time and memory consuming task. The memory used to simulation hydrogen chain in different size is shown in table~\ref{tab:memory_h_chain}

\begin{table}[ht]
    \begin{tabular}{ccc}
        \toprule
        Hydrogen chain & Qubit Number & Memory with Complex128 \\
        \midrule
        $H_2$          & 4            & 0.256kB                \\
        $H_6$          & 12           & 64kB                   \\
        $H_{10}$       & 20           & 16MB                   \\
        $H_{14}$       & 28           & 4GB                    \\
        $H_{18}$       & 36           & 1TB                    \\
        \bottomrule
    \end{tabular}
    \caption{Memory consuming for storage full amplitudes quantum state.}
    \label{tab:memory_h_chain}
\end{table}

Based on UCCSD (Unitary coupled-cluster with singles and doubles) theory, the ansatz is described as:

\begin{equation}
    U(\theta) = e^{(T-T^\dagger)}\ket{\psi_\text{HF}}
\end{equation}
where $T$ is couple cluster operator, and $\ket{\psi_\text{HF}}$ is hartree-fork state. The couple cluster operator is written as:
\begin{equation}
    T=\sum_{p\notin\text{occ},q\in\text{occ}}\theta_q^p a_p^\dagger a_q + \sum_{pq\notin\text{occ},rs\in\text{occ}}\theta_{rs}^{pq}a_p^\dagger a_q^\dagger a_r a_s,
\end{equation}
and under trotterization the ansatz can decompose to:
\begin{equation}
    U(\theta)=\prod_ie^{(T_i-T_i^\dagger)}\ket{\psi_\text{HF}}
\end{equation}
Taking hydrogen chain $H_4$ for example, we need 12 qubits to simulate this system and there are 6 electrons with 3 spin up and 3 spin down. Fig~\ref{6_1_h4_a} shows the hartree-fork state of $H_4$ and Fig~\ref{6_1_h4_b} shows the spin-orbital under excitation operator $T=a_{11}^\dagger a_{10}^\dagger a_5 a_4$.
\begin{figure}
    \centering
    \begin{subfigure}{0.3\textwidth}
        \centering
        \includegraphics[width=\textwidth]{images/6_1_h4_a.pdf}
        \caption{Hartree-fork state of $H_4$.}
        \label{6_1_h4_a}
    \end{subfigure}
    \begin{subfigure}{0.3\textwidth}
        \centering
        \includegraphics[width=0.9\textwidth]{images/6_1_h4_b.pdf}
        \caption{Spin-orbital after applying excitation operator $T=a_{11}^\dagger a_{10}^\dagger a_5 a_4$.}
        \label{6_1_h4_b}
    \end{subfigure}
\end{figure}

In \QuPack, we impose the constraint that the total number of electrons and the total spin number of electrons remain constant throughout the evolution of the ansatz. This constraint significantly reduces the dimension of the Hilbert space that we need to simulate. Suppose the qubit number is $n_q$ and the electron number is $n_e$, the dimension of full amplitude state is $2^{n_q}$, but after electron and spin conservation, the dimension will be:
\begin{equation}
    \binom{n_q/2}{n_e/2}^2=\left(\frac{(n_q/2)!}{(n_e/2)!(n_q/2-n_e/2)!}\right)^2.
\end{equation}
The memory consuming will reduce from 1TB to \~{}35GB.

Besides the memory reduction, we also optimized the evolution of coupled cluster operator. For a given coupled cluster operator $V(\theta)=e^{\theta(T-T^\dagger)}$, it is easy to prove that $(T-T^\dagger)^3 = - (T-T^\dagger)$ and $(T-T^\dagger)^4 = -(T-T^\dagger)^2$, after taylor expansion, we have:
\begin{equation}
    V(\theta) = \mathbb{I} + ( 1-\cos\theta)(T-T^\dagger)^2 + \sin\theta (T-T^\dagger).
\end{equation}
Choose a calculation base in reduced Hilbert space $\ket{i}$, so that $T\ket{i} = \ket{j}\neq0$, then we have $T^\dagger\ket{i}=0$ and $T^\dagger\ket{j}=\ket{i}$. The evolution of coupled cluster can simplified to:
\begin{equation}
    e^{\theta(T-T^\dagger)}\begin{pmatrix}
        \ket{i} \\\ket{j}
    \end{pmatrix}=\begin{pmatrix}
        \cos\theta  & \sin\theta \\
        -\sin\theta & \cos\theta
    \end{pmatrix}\begin{pmatrix}
        \ket{i} \\\ket{j}
    \end{pmatrix}
\end{equation}
And for $e^{i\theta(T+T^\dagger)}$, we have
\begin{equation}
    e^{i\theta(T+T^\dagger)}\begin{pmatrix}
        \ket{i} \\\ket{j}
    \end{pmatrix}=\begin{pmatrix}
        \cos\theta   & i\sin\theta \\
        -i\sin\theta & \cos\theta
    \end{pmatrix}\begin{pmatrix}
        \ket{i} \\\ket{j}
    \end{pmatrix}
\end{equation}

When doing variation quantum algorithm on this system, we need to calculate the gradient of $V(\theta)$, which is given as:
\begin{equation}
    \frac{\partial V(\theta)}{\partial\theta}\begin{pmatrix}
        \ket{i} \\\ket{j}
    \end{pmatrix}=\begin{pmatrix}
        -\sin\theta  & \cos\theta \\
        -\cos\theta & -\sin\theta
    \end{pmatrix}\begin{pmatrix}
        \ket{i} \\\ket{j}
    \end{pmatrix}.
\end{equation}
