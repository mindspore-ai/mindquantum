View demo code of this section: \democode{02}{2.4}

The quantum circuit is a graphical representation of the sequence of quantum gates applied to qubits in a quantum computation or quantum algorithm. Similar to classical digital circuits made up of logic gates that manipulate classical bits, quantum circuits are composed of quantum gates that manipulate qubits.

\subsubsection{Construct a quantum circuit}
In \MindQuantum\ we use \Circuit to represent quantum circuit:

\begin{lstlisting}
from mindquantum.core.circuit import Circuit
circ = Circuit()
\end{lstlisting}

There are several basic operations for adding a quantum gate into \Circuit
\begin{itemize}
    \item list: directly construct a quantum gate with gate list.
    \item \code{+=}: use \code{+=} to add a quantum gate to the circuit.
    \item \code{Circuit.x}: add an X gate to the circuit.
\end{itemize}
\begin{lstlisting}
from mindquantum.core.gates import H, Y, X, Z
circ = Circuit([X.on(0), Y.on(1)])
circ += H.on(0)
circ += Y.on(1, 0)
circ.x(1, 0)
circ.z(2, [0, 1])
\end{lstlisting}

the result is shown in Fig.~\ref{fig:Quantum-Circuit}.

\subsubsection{Display a quantum circuit}
SVG(Scalable Vector Graphics) is based on the XML markup language and is used to describe vector graphics in two dimensions. \MindQuantum\ provides a function to export quantum circuit to SVG format.
\begin{lstlisting}
circ.svg().to_file(filename='circuit.svg')
\end{lstlisting}
\begin{figure}[h]
    \begin{center}
        \includegraphics[width=0.9\linewidth]{images/2_4_circuit.png}
    \end{center}
    \caption{Quantum Circuit.}
    \label{fig:Quantum-Circuit}
\end{figure}
Please note that if you are in jupyter notebook environment, you can directly get SVG image by \code{circ.svg()}.


\subsubsection{Common interfaces}

Quantum Circuit is the basic element of quantum algorithm, \MindQuantum\ provide a lot of useful method for \Circuit.

\begin{itemize}
    \item \propnqubits: get the number of qubits of quantum circuit.
    \item \propparamsname: get the parameter names of circuit.
    \item \prophasmeasuregate: get whether to be measured.
    \item \methodmatrix: get the circuit matrix.
    \item \methodgetqs: get the final quantum state.
    \item \methodsummary: get information about the current circuit, including the number of blocks, gates, gates without parameters, gates with parameters and parameters.
\end{itemize}

\subsubsection{Advanced operator on circuit}

Constructing a large size quantum circuit is always not a straightforward thing for users. Here in \MindQuantum, we provide some pre-defined methods to make manipulating \Circuit more easily. For example, let the origin quantum circuit be:

\begin{lstlisting}
from mindquantum.core.circuit import *

circ = Circuit().h(0).rx('a', 2, 0)
\end{lstlisting}

\begin{figure}[H]
    \begin{center}
        \includegraphics[width=0.4\linewidth]{images/2_4_ori_circ.png}
    \end{center}
    \caption{Quantum Circuit need to manipulate.}
\end{figure}
We now show how to manipulate on this quantum circuit:


\begin{enumerate}
    \item \methodcompress{circ} : compress all qubit to first $n$ qubits.\par
          \begin{minipage}{\linewidth}
              \centering
              \includegraphics[width=0.4\linewidth]{images/2_4_compress_circ.png}
          \end{minipage}
    \item \methodcontrol{circ}{1} : add control qubits on this circuit\par
          \begin{minipage}{\linewidth}
              \centering
              \includegraphics[width=0.4\linewidth]{images/2_4_ctrl_circ.png}
          \end{minipage}
    \item \methoddagger{circ} : get hermitian conjugate version\par
          \begin{minipage}{\linewidth}
              \centering
              \includegraphics[width=0.4\linewidth]{images/2_4_dagger_circ.png}
          \end{minipage}
    \item \methodapply{circ}{[2, 1]} : apply circuit to other qubits\par
          \begin{minipage}{\linewidth}
              \centering
              \includegraphics[width=0.4\linewidth]{images/2_4_apply_circ.png}
          \end{minipage}
    \item \methodreverse{circ} : reverse the qubit order in circuit\par
          \begin{minipage}{\linewidth}
              \centering
              \includegraphics[width=0.4\linewidth]{images/2_4_reverse_circ.png}
          \end{minipage}
    \item \methodshift{circ}{2} : shift the qubit range with given step\par
          \begin{minipage}{\linewidth}
              \centering
              \includegraphics[width=0.4\linewidth]{images/2_4_shift_circ.png}
          \end{minipage}
\end{enumerate}
